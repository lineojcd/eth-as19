\documentclass[a4paper,12pt]{article}
\usepackage[utf8x]{inputenc}
\usepackage[T1]{fontenc}
%\usepackage{serohyph}
\usepackage[ngerman]{babel}
\usepackage{latexsym}
\usepackage[pdfpagelabels,plainpages=false]{hyperref} %Für Übersichts-Lesezeichen im PDF
			% Fuer Print besser ’false’
\usepackage{setspace} % Zeilenabstand ändern mi
%\usepackage{sectsty} % section style package
\usepackage[%small
%,compact
]{titlesec}  % Paket um Überschriften einfach zu manipulieren
\usepackage{enumitem} % Paket für Auflistungen zu manipulieren (\setlist{noitemsep})
\usepackage{color} % für Farben

%\titleformat{ command }[ shape ]{ format }{ label }{ sep }{ before-code }[ after-code ]
\titleformat{\section} % titlesec
  {\normalfont\bfseries} %
  {\bfseries \thesection}
  {.5em}
  {}
\titlespacing{\section} % Um angehenden Textblock nach \section layouten.
             {0pc}{2ex}{3ex}[0pc] % Hiermit wird der Abstand zwischen Titel und Text verkelinert (=0)
%{0pc}{2ex}{0ex}[0pc] Abstand zum linken Rand |Abstand zum oberen Text | Abstand zum unteren Text | Abstand zum rechten Rand
%ACHTUNG letztes Klammerpaar eckige Klammern!!
%
\hypersetup{		% von package hyperref
colorlinks=true,        % Links farbig ('true') oder nicht ('false'). Fuer
                        % Print besser 'false'
linkcolor=red,          % Farbe fuer die internen Links
urlcolor=blue,          % Farbe fuer externe Links (http://...)
pdfborder={0 0 0},      % wenn colorlinks nicht gesetzt ist gibt es einen
                        % Rand dieser Farbe (R, G, B} um den Link
                        % geoeffnet werden
pdfpagemode=UseOutlines,% andere Moeglichkeiten: 'None', 'UseThumbs'
                        % und 'FullScreen'
pdftitle={Titel des Dokuments}, pdfauthor={Autor(en)},
pdfsubject={Thema}, pdfkeywords={Keywords} }

\addtolength{\textwidth}{2cm} %Textblcok-Breite
\addtolength{\oddsidemargin}{-1cm} %Abst. 1 inch von linkem Rand
\addtolength{\headsep}{-2cm} %Abst zu Kopfzeile
\addtolength{\textheight}{3cm} % Textblock-Länge
%
\def\headtitle{Spatio-Temporal Modelling in Biology} % ToDo: Wiedergebbare Titel für den gesammten Text \headtitle
%
\title{\headtitle}
\author{VIS}
\makeindex

\begin{document}
\pagestyle{empty}
\parindent 0pt	% kein Erstzeileneinzug für das ganze Dokument
\flushleft % linksbündig
\setlist{noitemsep} %Kein Zeilenabstand zwischen Listen
%
%\maketitle
%
%\color{blue}
\color[rgb]{0,0,.5}
{\LARGE \headtitle\par}
\hrulefill\\
%
\section*{Oral Exam Report}
\vspace{.5cm}
%\begin{onehalfspace} %geht nur bei tabular, nicht bei "`tabbing"'
  \begin{tabbing}
    \quad \= Examiner:XXXX \= XXXXXXXXXXX \kill % Abstandgebende-Zeile
    \>Course:		\>	\headtitle\\[1ex]
    \>Examiner: 	\>	Iber, Dagmar, Prof. Dr.\\[1ex]	%ToDo
    \>Protocol: 	\>	Iber, Dagmar, Prof. Dr.\\[1ex]	%ToDo
    \>Semester:	\>	HS12\\[1ex]	%ToDo
    \>Datum: 		\>	2020-02-06%ToDo
  \end{tabbing}
%\end{onehalfspace}%
%
%\vspace{0.1cm}
\hrulefill\\
\color{black}
\vspace{.7cm}

%ToDo

\section{Setting}

I was seated in the office at Zentrum with the TA and Prof. Iber.

\section{First question}

Since I was the last one to take the course apparently, she gave me the option to choose a topic I like. I chose Turing pattern in biology.

I started describing the two component system (the easiest case) and briefly stated how causes the Turing pattern to emerge: the diffusion driven instability starting from noise. I also wrote down the derivation of all the conditions involving $det(J) < 0, tr(J) > 0$ and $D_1, D_2 > 0$ and different and another condition. The latter is the condition for instability. And I also wrote why we need $det(H) < 0$ as well as how it connects to wave number $k$ and $n$. 

In-between the writing, she would asked about the relationship between the wave number and the patterning in the end, which is how many modes you would get finally.

She also asked me to draw in the graph of $\mathrm{Re}(\lambda)$ versus $k^2$. I got the direction wrong initially but corrected it after she gave a hint and re-checked the equation. It is easier to remember by knowing when $k^2 = 0$, without diffusion, would resulted in a stable solution and thus negative real part of eigen-values.

\section{Second question Problem with scaling domain}

Prof. asked me what happens in general in case of growing domain. I answered it with a specific model: the standard morphogen gradient model with constant degradation. I wrote down the formula and show how the difference in source can affect the read-out: $\Delta x = \lambda \ln{(c_0^*/c_0)}$. She then further asked me how would this steady state gradient be affected by this result. I didn't quite get the idea in the first place. But I realized that I could use the plot of $ln(c_0)$ versus $x$. If the domain length grows and under the french flag model, the read-out sensed by the receptors would not scale properly with a linearly growing domain. She also asked about how would one deal with a growling domain and how the coefficients cope with that. Due to a lack of time, I just wrote down the equation $\frac{\partial}{\partial t} c + \nabla(c\cdot u) = R(c) + D \Delta c$ and how to keep lambda fixed by letting $D/k \propto L^2$

\end{document}
